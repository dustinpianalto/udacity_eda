\documentclass[]{article}
\usepackage{lmodern}
\usepackage{amssymb,amsmath}
\usepackage{ifxetex,ifluatex}
\usepackage{fixltx2e} % provides \textsubscript
\ifnum 0\ifxetex 1\fi\ifluatex 1\fi=0 % if pdftex
  \usepackage[T1]{fontenc}
  \usepackage[utf8]{inputenc}
\else % if luatex or xelatex
  \ifxetex
    \usepackage{mathspec}
  \else
    \usepackage{fontspec}
  \fi
  \defaultfontfeatures{Ligatures=TeX,Scale=MatchLowercase}
\fi
% use upquote if available, for straight quotes in verbatim environments
\IfFileExists{upquote.sty}{\usepackage{upquote}}{}
% use microtype if available
\IfFileExists{microtype.sty}{%
\usepackage{microtype}
\UseMicrotypeSet[protrusion]{basicmath} % disable protrusion for tt fonts
}{}
\usepackage[margin=1in]{geometry}
\usepackage{hyperref}
\hypersetup{unicode=true,
            pdfborder={0 0 0},
            breaklinks=true}
\urlstyle{same}  % don't use monospace font for urls
\usepackage{color}
\usepackage{fancyvrb}
\newcommand{\VerbBar}{|}
\newcommand{\VERB}{\Verb[commandchars=\\\{\}]}
\DefineVerbatimEnvironment{Highlighting}{Verbatim}{commandchars=\\\{\}}
% Add ',fontsize=\small' for more characters per line
\usepackage{framed}
\definecolor{shadecolor}{RGB}{248,248,248}
\newenvironment{Shaded}{\begin{snugshade}}{\end{snugshade}}
\newcommand{\KeywordTok}[1]{\textcolor[rgb]{0.13,0.29,0.53}{\textbf{#1}}}
\newcommand{\DataTypeTok}[1]{\textcolor[rgb]{0.13,0.29,0.53}{#1}}
\newcommand{\DecValTok}[1]{\textcolor[rgb]{0.00,0.00,0.81}{#1}}
\newcommand{\BaseNTok}[1]{\textcolor[rgb]{0.00,0.00,0.81}{#1}}
\newcommand{\FloatTok}[1]{\textcolor[rgb]{0.00,0.00,0.81}{#1}}
\newcommand{\ConstantTok}[1]{\textcolor[rgb]{0.00,0.00,0.00}{#1}}
\newcommand{\CharTok}[1]{\textcolor[rgb]{0.31,0.60,0.02}{#1}}
\newcommand{\SpecialCharTok}[1]{\textcolor[rgb]{0.00,0.00,0.00}{#1}}
\newcommand{\StringTok}[1]{\textcolor[rgb]{0.31,0.60,0.02}{#1}}
\newcommand{\VerbatimStringTok}[1]{\textcolor[rgb]{0.31,0.60,0.02}{#1}}
\newcommand{\SpecialStringTok}[1]{\textcolor[rgb]{0.31,0.60,0.02}{#1}}
\newcommand{\ImportTok}[1]{#1}
\newcommand{\CommentTok}[1]{\textcolor[rgb]{0.56,0.35,0.01}{\textit{#1}}}
\newcommand{\DocumentationTok}[1]{\textcolor[rgb]{0.56,0.35,0.01}{\textbf{\textit{#1}}}}
\newcommand{\AnnotationTok}[1]{\textcolor[rgb]{0.56,0.35,0.01}{\textbf{\textit{#1}}}}
\newcommand{\CommentVarTok}[1]{\textcolor[rgb]{0.56,0.35,0.01}{\textbf{\textit{#1}}}}
\newcommand{\OtherTok}[1]{\textcolor[rgb]{0.56,0.35,0.01}{#1}}
\newcommand{\FunctionTok}[1]{\textcolor[rgb]{0.00,0.00,0.00}{#1}}
\newcommand{\VariableTok}[1]{\textcolor[rgb]{0.00,0.00,0.00}{#1}}
\newcommand{\ControlFlowTok}[1]{\textcolor[rgb]{0.13,0.29,0.53}{\textbf{#1}}}
\newcommand{\OperatorTok}[1]{\textcolor[rgb]{0.81,0.36,0.00}{\textbf{#1}}}
\newcommand{\BuiltInTok}[1]{#1}
\newcommand{\ExtensionTok}[1]{#1}
\newcommand{\PreprocessorTok}[1]{\textcolor[rgb]{0.56,0.35,0.01}{\textit{#1}}}
\newcommand{\AttributeTok}[1]{\textcolor[rgb]{0.77,0.63,0.00}{#1}}
\newcommand{\RegionMarkerTok}[1]{#1}
\newcommand{\InformationTok}[1]{\textcolor[rgb]{0.56,0.35,0.01}{\textbf{\textit{#1}}}}
\newcommand{\WarningTok}[1]{\textcolor[rgb]{0.56,0.35,0.01}{\textbf{\textit{#1}}}}
\newcommand{\AlertTok}[1]{\textcolor[rgb]{0.94,0.16,0.16}{#1}}
\newcommand{\ErrorTok}[1]{\textcolor[rgb]{0.64,0.00,0.00}{\textbf{#1}}}
\newcommand{\NormalTok}[1]{#1}
\usepackage{graphicx,grffile}
\makeatletter
\def\maxwidth{\ifdim\Gin@nat@width>\linewidth\linewidth\else\Gin@nat@width\fi}
\def\maxheight{\ifdim\Gin@nat@height>\textheight\textheight\else\Gin@nat@height\fi}
\makeatother
% Scale images if necessary, so that they will not overflow the page
% margins by default, and it is still possible to overwrite the defaults
% using explicit options in \includegraphics[width, height, ...]{}
\setkeys{Gin}{width=\maxwidth,height=\maxheight,keepaspectratio}
\IfFileExists{parskip.sty}{%
\usepackage{parskip}
}{% else
\setlength{\parindent}{0pt}
\setlength{\parskip}{6pt plus 2pt minus 1pt}
}
\setlength{\emergencystretch}{3em}  % prevent overfull lines
\providecommand{\tightlist}{%
  \setlength{\itemsep}{0pt}\setlength{\parskip}{0pt}}
\setcounter{secnumdepth}{0}
% Redefines (sub)paragraphs to behave more like sections
\ifx\paragraph\undefined\else
\let\oldparagraph\paragraph
\renewcommand{\paragraph}[1]{\oldparagraph{#1}\mbox{}}
\fi
\ifx\subparagraph\undefined\else
\let\oldsubparagraph\subparagraph
\renewcommand{\subparagraph}[1]{\oldsubparagraph{#1}\mbox{}}
\fi

%%% Use protect on footnotes to avoid problems with footnotes in titles
\let\rmarkdownfootnote\footnote%
\def\footnote{\protect\rmarkdownfootnote}

%%% Change title format to be more compact
\usepackage{titling}

% Create subtitle command for use in maketitle
\newcommand{\subtitle}[1]{
  \posttitle{
    \begin{center}\large#1\end{center}
    }
}

\setlength{\droptitle}{-2em}
  \title{}
  \pretitle{\vspace{\droptitle}}
  \posttitle{}
  \author{}
  \preauthor{}\postauthor{}
  \date{}
  \predate{}\postdate{}


\begin{document}

\section{Lesson 4}\label{lesson-4}

\begin{center}\rule{0.5\linewidth}{\linethickness}\end{center}

\subsubsection{Scatterplots and Perceived Audience
Size}\label{scatterplots-and-perceived-audience-size}

Notes:

\begin{center}\rule{0.5\linewidth}{\linethickness}\end{center}

\subsubsection{Scatterplots}\label{scatterplots}

Notes:

\begin{Shaded}
\begin{Highlighting}[]
\KeywordTok{library}\NormalTok{(ggplot2)}
\NormalTok{pf <-}\StringTok{ }\KeywordTok{read.csv}\NormalTok{(}\StringTok{'pseudo_facebook.tsv'}\NormalTok{, }\DataTypeTok{sep =} \StringTok{'}\CharTok{\textbackslash{}t}\StringTok{'}\NormalTok{)}

\KeywordTok{ggplot}\NormalTok{(}\KeywordTok{aes}\NormalTok{(}\DataTypeTok{x =}\NormalTok{ age, }\DataTypeTok{y =}\NormalTok{ friend_count), }\DataTypeTok{data =}\NormalTok{ pf) }\OperatorTok{+}
\StringTok{  }\KeywordTok{geom_point}\NormalTok{()}
\end{Highlighting}
\end{Shaded}

\includegraphics{lesson4_student_files/figure-latex/Scatterplots-1.pdf}

\begin{center}\rule{0.5\linewidth}{\linethickness}\end{center}

\paragraph{What are some things that you notice right
away?}\label{what-are-some-things-that-you-notice-right-away}

Response:

All of the data points are grouped into vertical lines and that the
younger the age the more likely they are to have more friends.

\subsubsection{ggplot Syntax}\label{ggplot-syntax}

Notes:

\begin{Shaded}
\begin{Highlighting}[]
\KeywordTok{ggplot}\NormalTok{(}\KeywordTok{aes}\NormalTok{(}\DataTypeTok{x =}\NormalTok{ age, }\DataTypeTok{y =}\NormalTok{ friend_count), }\DataTypeTok{data =}\NormalTok{ pf) }\OperatorTok{+}
\StringTok{  }\KeywordTok{geom_point}\NormalTok{() }\OperatorTok{+}
\StringTok{  }\KeywordTok{xlim}\NormalTok{(}\DecValTok{13}\NormalTok{, }\DecValTok{90}\NormalTok{)}
\end{Highlighting}
\end{Shaded}

\begin{verbatim}
## Warning: Removed 4906 rows containing missing values (geom_point).
\end{verbatim}

\includegraphics{lesson4_student_files/figure-latex/ggplot Syntax-1.pdf}

\begin{Shaded}
\begin{Highlighting}[]
\KeywordTok{summary}\NormalTok{(pf}\OperatorTok{$}\NormalTok{age)}
\end{Highlighting}
\end{Shaded}

\begin{verbatim}
##    Min. 1st Qu.  Median    Mean 3rd Qu.    Max. 
##   13.00   20.00   28.00   37.28   50.00  113.00
\end{verbatim}

Build one layer at a time to find errors easier

\subsubsection{Overplotting}\label{overplotting}

Notes:

\begin{Shaded}
\begin{Highlighting}[]
\KeywordTok{ggplot}\NormalTok{(}\KeywordTok{aes}\NormalTok{(}\DataTypeTok{x =}\NormalTok{ age, }\DataTypeTok{y =}\NormalTok{ friend_count), }\DataTypeTok{data =}\NormalTok{ pf) }\OperatorTok{+}
\StringTok{  }\KeywordTok{geom_jitter}\NormalTok{(}\DataTypeTok{alpha =} \DecValTok{1}\OperatorTok{/}\DecValTok{20}\NormalTok{) }\OperatorTok{+}
\StringTok{  }\KeywordTok{xlim}\NormalTok{(}\DecValTok{13}\NormalTok{, }\DecValTok{90}\NormalTok{)}
\end{Highlighting}
\end{Shaded}

\begin{verbatim}
## Warning: Removed 5190 rows containing missing values (geom_point).
\end{verbatim}

\includegraphics{lesson4_student_files/figure-latex/Overplotting-1.pdf}

\paragraph{What do you notice in the
plot?}\label{what-do-you-notice-in-the-plot}

Response:

The bar for 69 is still clearly visible and it is more obvious that the
number generally decreases as the age increases.

\subsubsection{Coord\_trans()}\label{coord_trans}

Notes:

\paragraph{Look up the documentation for coord\_trans() and add a layer
to the plot that transforms friend\_count using the square root
function. Create your
plot!}\label{look-up-the-documentation-for-coord_trans-and-add-a-layer-to-the-plot-that-transforms-friend_count-using-the-square-root-function.-create-your-plot}

\begin{Shaded}
\begin{Highlighting}[]
\KeywordTok{ggplot}\NormalTok{(}\KeywordTok{aes}\NormalTok{(}\DataTypeTok{x =}\NormalTok{ age, }\DataTypeTok{y =}\NormalTok{ friend_count), }\DataTypeTok{data =}\NormalTok{ pf) }\OperatorTok{+}
\StringTok{  }\KeywordTok{geom_point}\NormalTok{(}\DataTypeTok{alpha =} \DecValTok{1}\OperatorTok{/}\DecValTok{20}\NormalTok{) }\OperatorTok{+}
\StringTok{  }\KeywordTok{xlim}\NormalTok{(}\DecValTok{13}\NormalTok{, }\DecValTok{90}\NormalTok{) }\OperatorTok{+}
\StringTok{  }\KeywordTok{coord_trans}\NormalTok{(}\DataTypeTok{y =} \StringTok{"sqrt"}\NormalTok{)}
\end{Highlighting}
\end{Shaded}

\begin{verbatim}
## Warning: Removed 4906 rows containing missing values (geom_point).
\end{verbatim}

\includegraphics{lesson4_student_files/figure-latex/unnamed-chunk-1-1.pdf}

\paragraph{What do you notice?}\label{what-do-you-notice}

First off coord\_trans does not work with geom\_jitter, second the
datapoints near the bottom are more spread out vertically to present
them as more of a focus.

To use jitter you need more advanced syntax to only jitter the ages,
also to prevent possible negatives if 0 is jittered. To do this in
\texttt{geom\_point()} pass
\texttt{position\ =\ position\_jitter(h\ =\ 0)}

\begin{Shaded}
\begin{Highlighting}[]
\KeywordTok{ggplot}\NormalTok{(}\KeywordTok{aes}\NormalTok{(}\DataTypeTok{x =}\NormalTok{ age, }\DataTypeTok{y =}\NormalTok{ friend_count), }\DataTypeTok{data =}\NormalTok{ pf) }\OperatorTok{+}
\StringTok{  }\KeywordTok{geom_point}\NormalTok{(}\DataTypeTok{alpha =} \DecValTok{1}\OperatorTok{/}\DecValTok{20}\NormalTok{, }\DataTypeTok{position =} \KeywordTok{position_jitter}\NormalTok{(}\DataTypeTok{h =} \DecValTok{0}\NormalTok{)) }\OperatorTok{+}
\StringTok{  }\KeywordTok{xlim}\NormalTok{(}\DecValTok{13}\NormalTok{, }\DecValTok{90}\NormalTok{) }\OperatorTok{+}
\StringTok{  }\KeywordTok{coord_trans}\NormalTok{(}\DataTypeTok{y =} \StringTok{"sqrt"}\NormalTok{)}
\end{Highlighting}
\end{Shaded}

\begin{verbatim}
## Warning: Removed 5185 rows containing missing values (geom_point).
\end{verbatim}

\includegraphics{lesson4_student_files/figure-latex/coord_trans_advanced-1.pdf}

\subsubsection{Alpha and Jitter}\label{alpha-and-jitter}

Notes:

\begin{Shaded}
\begin{Highlighting}[]
\KeywordTok{ggplot}\NormalTok{(}\KeywordTok{aes}\NormalTok{(}\DataTypeTok{x =}\NormalTok{ age, }\DataTypeTok{y =}\NormalTok{ friendships_initiated, }\DataTypeTok{color =}\NormalTok{ gender), }\DataTypeTok{data =}\NormalTok{ pf) }\OperatorTok{+}
\StringTok{  }\KeywordTok{geom_point}\NormalTok{(}\DataTypeTok{alpha =} \DecValTok{1}\OperatorTok{/}\DecValTok{10}\NormalTok{, }\DataTypeTok{position =} \KeywordTok{position_jitter}\NormalTok{(}\DataTypeTok{h =} \DecValTok{0}\NormalTok{)) }\OperatorTok{+}
\StringTok{  }\KeywordTok{xlim}\NormalTok{(}\DecValTok{13}\NormalTok{, }\DecValTok{90}\NormalTok{) }\OperatorTok{+}
\StringTok{  }\KeywordTok{coord_trans}\NormalTok{(}\DataTypeTok{y =} \StringTok{"sqrt"}\NormalTok{)}
\end{Highlighting}
\end{Shaded}

\begin{verbatim}
## Warning: Removed 5170 rows containing missing values (geom_point).
\end{verbatim}

\includegraphics{lesson4_student_files/figure-latex/Alpha and Jitter-1.pdf}

\begin{center}\rule{0.5\linewidth}{\linethickness}\end{center}

\subsubsection{Overplotting and Domain
Knowledge}\label{overplotting-and-domain-knowledge}

Notes:

plotting as a percentage of the whole

\subsubsection{Conditional Means}\label{conditional-means}

Notes:

\begin{Shaded}
\begin{Highlighting}[]
\KeywordTok{library}\NormalTok{(dplyr)}
\end{Highlighting}
\end{Shaded}

\begin{verbatim}
## 
## Attaching package: 'dplyr'
\end{verbatim}

\begin{verbatim}
## The following objects are masked from 'package:stats':
## 
##     filter, lag
\end{verbatim}

\begin{verbatim}
## The following objects are masked from 'package:base':
## 
##     intersect, setdiff, setequal, union
\end{verbatim}

\begin{Shaded}
\begin{Highlighting}[]
\NormalTok{age_groups <-}\StringTok{ }\KeywordTok{group_by}\NormalTok{(pf, age)}
\NormalTok{pf.fc_by_age <-}\StringTok{ }\KeywordTok{summarise}\NormalTok{(age_groups,}
          \DataTypeTok{friend_count_mean =} \KeywordTok{mean}\NormalTok{(friend_count),}
          \DataTypeTok{friend_count_median =} \KeywordTok{median}\NormalTok{(friend_count),}
          \DataTypeTok{n =} \KeywordTok{n}\NormalTok{())}
\NormalTok{pf.fc_by_age <-}\StringTok{ }\KeywordTok{arrange}\NormalTok{(pf.fc_by_age, age)}

\KeywordTok{ggplot}\NormalTok{(}\KeywordTok{aes}\NormalTok{(}\DataTypeTok{x =}\NormalTok{ age, }\DataTypeTok{y =}\NormalTok{ friend_count_mean), }\DataTypeTok{data =}\NormalTok{ pf.fc_by_age) }\OperatorTok{+}
\StringTok{  }\KeywordTok{geom_line}\NormalTok{() }\OperatorTok{+}
\StringTok{  }\KeywordTok{xlim}\NormalTok{(}\DecValTok{13}\NormalTok{,}\DecValTok{90}\NormalTok{)}
\end{Highlighting}
\end{Shaded}

\begin{verbatim}
## Warning: Removed 23 rows containing missing values (geom_path).
\end{verbatim}

\includegraphics{lesson4_student_files/figure-latex/Conditional Means-1.pdf}

\subsubsection{Overlaying Summaries with Raw
Data}\label{overlaying-summaries-with-raw-data}

Notes:

\begin{Shaded}
\begin{Highlighting}[]
\KeywordTok{ggplot}\NormalTok{(}\KeywordTok{aes}\NormalTok{(}\DataTypeTok{x =}\NormalTok{ age, }\DataTypeTok{y =}\NormalTok{ friendships_initiated), }\DataTypeTok{data =}\NormalTok{ pf) }\OperatorTok{+}
\StringTok{  }\KeywordTok{geom_point}\NormalTok{(}\DataTypeTok{alpha =} \DecValTok{1}\OperatorTok{/}\DecValTok{10}\NormalTok{, }\DataTypeTok{position =} \KeywordTok{position_jitter}\NormalTok{(}\DataTypeTok{h =} \DecValTok{0}\NormalTok{), }\DataTypeTok{color =} \StringTok{'orange'}\NormalTok{) }\OperatorTok{+}
\StringTok{  }\KeywordTok{xlim}\NormalTok{(}\DecValTok{13}\NormalTok{, }\DecValTok{90}\NormalTok{) }\OperatorTok{+}
\StringTok{  }\KeywordTok{coord_trans}\NormalTok{(}\DataTypeTok{y =} \StringTok{"sqrt"}\NormalTok{) }\OperatorTok{+}
\StringTok{  }\KeywordTok{geom_line}\NormalTok{(}\DataTypeTok{stat =} \StringTok{'summary'}\NormalTok{, }\DataTypeTok{fun.y =}\NormalTok{ mean) }\OperatorTok{+}
\StringTok{  }\KeywordTok{geom_line}\NormalTok{(}\DataTypeTok{stat =} \StringTok{'summary'}\NormalTok{, }\DataTypeTok{fun.y =}\NormalTok{ median, }\DataTypeTok{color =} \StringTok{'blue'}\NormalTok{) }\OperatorTok{+}
\StringTok{  }\KeywordTok{geom_line}\NormalTok{(}\DataTypeTok{stat =} \StringTok{'summary'}\NormalTok{, }\DataTypeTok{fun.y =}\NormalTok{ quantile, }\DataTypeTok{fun.args =} \KeywordTok{list}\NormalTok{(}\DataTypeTok{probs =} \FloatTok{0.1}\NormalTok{), }\DataTypeTok{color =} \StringTok{'red'}\NormalTok{, }\DataTypeTok{linetype =} \DecValTok{2}\NormalTok{) }\OperatorTok{+}
\StringTok{  }\KeywordTok{geom_line}\NormalTok{(}\DataTypeTok{stat =} \StringTok{'summary'}\NormalTok{, }\DataTypeTok{fun.y =}\NormalTok{ quantile, }\DataTypeTok{fun.args =} \KeywordTok{list}\NormalTok{(}\DataTypeTok{probs =} \FloatTok{0.9}\NormalTok{), }\DataTypeTok{color =} \StringTok{'red'}\NormalTok{, }\DataTypeTok{linetype =} \DecValTok{2}\NormalTok{) }\OperatorTok{+}
\StringTok{  }\KeywordTok{coord_cartesian}\NormalTok{(}\DataTypeTok{xlim =} \KeywordTok{c}\NormalTok{(}\DecValTok{13}\NormalTok{,}\DecValTok{70}\NormalTok{), }\DataTypeTok{ylim =} \KeywordTok{c}\NormalTok{(}\DecValTok{0}\NormalTok{,}\DecValTok{1000}\NormalTok{))}
\end{Highlighting}
\end{Shaded}

\begin{verbatim}
## Warning: Removed 4906 rows containing non-finite values (stat_summary).

## Warning: Removed 4906 rows containing non-finite values (stat_summary).

## Warning: Removed 4906 rows containing non-finite values (stat_summary).

## Warning: Removed 4906 rows containing non-finite values (stat_summary).
\end{verbatim}

\begin{verbatim}
## Warning: Removed 5183 rows containing missing values (geom_point).
\end{verbatim}

\includegraphics{lesson4_student_files/figure-latex/Overlaying Summaries with Raw Data-1.pdf}

\paragraph{What are some of your observations of the
plot?}\label{what-are-some-of-your-observations-of-the-plot}

Response:

I notice that the median is always lower than the mean and that the
median is closer to the center of the main body of datapoints. It
appears that the data is long tailed towards the high friend counts
which pulls the mean upwards.

\subsubsection{Moira: Histogram Summary and
Scatterplot}\label{moira-histogram-summary-and-scatterplot}

See the Instructor Notes of this video to download Moira's paper on
perceived audience size and to see the final plot.

Notes:

\begin{center}\rule{0.5\linewidth}{\linethickness}\end{center}

\subsubsection{Correlation}\label{correlation}

Notes:

\begin{Shaded}
\begin{Highlighting}[]
\KeywordTok{cor.test}\NormalTok{(pf}\OperatorTok{$}\NormalTok{age, pf}\OperatorTok{$}\NormalTok{friend_count)}
\end{Highlighting}
\end{Shaded}

\begin{verbatim}
## 
##  Pearson's product-moment correlation
## 
## data:  pf$age and pf$friend_count
## t = -8.6268, df = 99001, p-value < 2.2e-16
## alternative hypothesis: true correlation is not equal to 0
## 95 percent confidence interval:
##  -0.03363072 -0.02118189
## sample estimates:
##         cor 
## -0.02740737
\end{verbatim}

Look up the documentation for the cor.test function.

What's the correlation between age and friend count? Round to three
decimal places. Response:

-0.027

\subsubsection{Correlation on Subsets}\label{correlation-on-subsets}

Notes:

\begin{Shaded}
\begin{Highlighting}[]
\KeywordTok{with}\NormalTok{(pf[pf}\OperatorTok{$}\NormalTok{age }\OperatorTok{<=}\StringTok{ }\DecValTok{70}\NormalTok{,], }\KeywordTok{cor.test}\NormalTok{(age, friend_count))}
\end{Highlighting}
\end{Shaded}

\begin{verbatim}
## 
##  Pearson's product-moment correlation
## 
## data:  age and friend_count
## t = -52.592, df = 91029, p-value < 2.2e-16
## alternative hypothesis: true correlation is not equal to 0
## 95 percent confidence interval:
##  -0.1780220 -0.1654129
## sample estimates:
##        cor 
## -0.1717245
\end{verbatim}

\begin{center}\rule{0.5\linewidth}{\linethickness}\end{center}

\subsubsection{Correlation Methods}\label{correlation-methods}

Notes:

\url{http://www.statisticssolutions.com/correlation-pearson-kendall-spearman/}

\subsection{Create Scatterplots}\label{create-scatterplots}

Notes:

\begin{Shaded}
\begin{Highlighting}[]
\KeywordTok{library}\NormalTok{(ggplot2)}
\KeywordTok{ggplot}\NormalTok{(}\KeywordTok{aes}\NormalTok{(}\DataTypeTok{x =}\NormalTok{ www_likes_received, }\DataTypeTok{y =}\NormalTok{ likes_received), }\DataTypeTok{data =}\NormalTok{ pf) }\OperatorTok{+}
\StringTok{  }\KeywordTok{geom_point}\NormalTok{()}\CommentTok{#alpha = 1/20, position = position_jitter(h = 0)) +}
\end{Highlighting}
\end{Shaded}

\includegraphics{lesson4_student_files/figure-latex/unnamed-chunk-2-1.pdf}

\begin{Shaded}
\begin{Highlighting}[]
  \CommentTok{#xlim(13, 90) +}
  \CommentTok{#coord_trans(y = "sqrt")}
\end{Highlighting}
\end{Shaded}

\begin{center}\rule{0.5\linewidth}{\linethickness}\end{center}

\subsubsection{Strong Correlations}\label{strong-correlations}

Notes:

\begin{Shaded}
\begin{Highlighting}[]
\KeywordTok{ggplot}\NormalTok{(}\KeywordTok{aes}\NormalTok{(}\DataTypeTok{x =}\NormalTok{ www_likes_received, }\DataTypeTok{y =}\NormalTok{ likes_received), }\DataTypeTok{data =}\NormalTok{ pf) }\OperatorTok{+}
\StringTok{  }\KeywordTok{geom_point}\NormalTok{() }\OperatorTok{+}
\StringTok{  }\KeywordTok{xlim}\NormalTok{(}\DecValTok{0}\NormalTok{, }\KeywordTok{quantile}\NormalTok{(pf}\OperatorTok{$}\NormalTok{www_likes_received, }\FloatTok{0.95}\NormalTok{)) }\OperatorTok{+}
\StringTok{  }\KeywordTok{ylim}\NormalTok{(}\DecValTok{0}\NormalTok{, }\KeywordTok{quantile}\NormalTok{(pf}\OperatorTok{$}\NormalTok{likes_received, }\FloatTok{0.95}\NormalTok{)) }\OperatorTok{+}
\StringTok{  }\KeywordTok{geom_smooth}\NormalTok{(}\DataTypeTok{method =} \StringTok{'lm'}\NormalTok{, }\DataTypeTok{color =} \StringTok{'red'}\NormalTok{)}
\end{Highlighting}
\end{Shaded}

\begin{verbatim}
## Warning: Removed 6075 rows containing non-finite values (stat_smooth).
\end{verbatim}

\begin{verbatim}
## Warning: Removed 6075 rows containing missing values (geom_point).
\end{verbatim}

\includegraphics{lesson4_student_files/figure-latex/Strong Correlations-1.pdf}

What's the correlation betwen the two variables? Include the top 5\% of
values for the variable in the calculation and round to 3 decimal
places.

\begin{Shaded}
\begin{Highlighting}[]
\KeywordTok{with}\NormalTok{(pf, }\KeywordTok{cor.test}\NormalTok{(www_likes_received, likes_received))}
\end{Highlighting}
\end{Shaded}

\begin{verbatim}
## 
##  Pearson's product-moment correlation
## 
## data:  www_likes_received and likes_received
## t = 937.1, df = 99001, p-value < 2.2e-16
## alternative hypothesis: true correlation is not equal to 0
## 95 percent confidence interval:
##  0.9473553 0.9486176
## sample estimates:
##       cor 
## 0.9479902
\end{verbatim}

Response:

0.948 Variable is a superset of another

\subsubsection{Moira on Correlation}\label{moira-on-correlation}

Notes:

Highly corelated can mean that variables are dependent on the same thing
or are similar.

\subsubsection{More Caution with
Correlation}\label{more-caution-with-correlation}

Notes:

\begin{Shaded}
\begin{Highlighting}[]
\CommentTok{#install.packages('alr3')}
\KeywordTok{library}\NormalTok{(alr3)}
\end{Highlighting}
\end{Shaded}

\begin{verbatim}
## Loading required package: car
\end{verbatim}

\begin{verbatim}
## Loading required package: carData
\end{verbatim}

\begin{verbatim}
## 
## Attaching package: 'car'
\end{verbatim}

\begin{verbatim}
## The following object is masked from 'package:dplyr':
## 
##     recode
\end{verbatim}

\begin{Shaded}
\begin{Highlighting}[]
\KeywordTok{library}\NormalTok{(ggplot2)}
\KeywordTok{data}\NormalTok{(Mitchell)}
\KeywordTok{ggplot}\NormalTok{(}\KeywordTok{aes}\NormalTok{(}\DataTypeTok{x =}\NormalTok{ Month, }\DataTypeTok{y =}\NormalTok{ Temp), }\DataTypeTok{data =}\NormalTok{ Mitchell) }\OperatorTok{+}
\StringTok{  }\KeywordTok{geom_point}\NormalTok{()}
\end{Highlighting}
\end{Shaded}

\includegraphics{lesson4_student_files/figure-latex/More Caution With Correlation-1.pdf}

Create your plot!

\begin{center}\rule{0.5\linewidth}{\linethickness}\end{center}

\subsubsection{Noisy Scatterplots}\label{noisy-scatterplots}

\begin{enumerate}
\def\labelenumi{\alph{enumi}.}
\tightlist
\item
  Take a guess for the correlation coefficient for the scatterplot.
\end{enumerate}

0.9

\begin{enumerate}
\def\labelenumi{\alph{enumi}.}
\setcounter{enumi}{1}
\tightlist
\item
  What is the actual correlation of the two variables? (Round to the
  thousandths place)
\end{enumerate}

\begin{Shaded}
\begin{Highlighting}[]
\KeywordTok{with}\NormalTok{(Mitchell, }\KeywordTok{cor.test}\NormalTok{(Month, Temp))}
\end{Highlighting}
\end{Shaded}

\begin{verbatim}
## 
##  Pearson's product-moment correlation
## 
## data:  Month and Temp
## t = 0.81816, df = 202, p-value = 0.4142
## alternative hypothesis: true correlation is not equal to 0
## 95 percent confidence interval:
##  -0.08053637  0.19331562
## sample estimates:
##        cor 
## 0.05747063
\end{verbatim}

\begin{center}\rule{0.5\linewidth}{\linethickness}\end{center}

\subsubsection{Making Sense of Data}\label{making-sense-of-data}

Notes:

\begin{Shaded}
\begin{Highlighting}[]
\KeywordTok{ggplot}\NormalTok{(}\KeywordTok{aes}\NormalTok{(Month, Temp), }\DataTypeTok{data =}\NormalTok{ Mitchell) }\OperatorTok{+}
\StringTok{  }\KeywordTok{geom_point}\NormalTok{() }\OperatorTok{+}
\StringTok{  }\KeywordTok{scale_x_continuous}\NormalTok{(}\DataTypeTok{breaks =} \KeywordTok{seq}\NormalTok{(}\DecValTok{0}\NormalTok{, }\DecValTok{204}\NormalTok{, }\DecValTok{12}\NormalTok{))}
\end{Highlighting}
\end{Shaded}

\includegraphics{lesson4_student_files/figure-latex/Making Sense of Data-1.pdf}

\begin{center}\rule{0.5\linewidth}{\linethickness}\end{center}

\subsubsection{A New Perspective}\label{a-new-perspective}

What do you notice? Response:

There is a cyclical pattern to the data going from low to high and back
to low every 12 months. This is why I originally said there seems to be
a 0.9 correlation coefficient to the data because I saw this pattern the
first time I looked at the plot.

Watch the solution video and check out the Instructor Notes! Notes:

\begin{Shaded}
\begin{Highlighting}[]
\KeywordTok{ggplot}\NormalTok{(}\KeywordTok{aes}\NormalTok{(}\DataTypeTok{x =}\NormalTok{ (Month}\OperatorTok\DecValTok{12}\NormalTok{), }\DataTypeTok{y =}\NormalTok{ Temp), }\DataTypeTok{data =}\NormalTok{ Mitchell) }\OperatorTok{+}
\StringTok{  }\KeywordTok{geom_point}\NormalTok{()}
\end{Highlighting}
\end{Shaded}

\includegraphics{lesson4_student_files/figure-latex/unnamed-chunk-3-1.pdf}

\subsubsection{Understanding Noise: Age to Age
Months}\label{understanding-noise-age-to-age-months}

Notes:

\begin{Shaded}
\begin{Highlighting}[]
\NormalTok{pf}\OperatorTok{$}\NormalTok{age_with_months <-}\StringTok{ }\NormalTok{(pf}\OperatorTok{$}\NormalTok{age) }\OperatorTok{+}\StringTok{ }\NormalTok{(}\DecValTok{1} \OperatorTok{-}\StringTok{ }\NormalTok{(pf}\OperatorTok{$}\NormalTok{dob_month}\OperatorTok{/}\DecValTok{12}\NormalTok{))}
\KeywordTok{head}\NormalTok{(pf)}
\end{Highlighting}
\end{Shaded}

\begin{verbatim}
##    userid age dob_day dob_year dob_month gender tenure friend_count
## 1 2094382  14      19     1999        11   male    266            0
## 2 1192601  14       2     1999        11 female      6            0
## 3 2083884  14      16     1999        11   male     13            0
## 4 1203168  14      25     1999        12 female     93            0
## 5 1733186  14       4     1999        12   male     82            0
## 6 1524765  14       1     1999        12   male     15            0
##   friendships_initiated likes likes_received mobile_likes
## 1                     0     0              0            0
## 2                     0     0              0            0
## 3                     0     0              0            0
## 4                     0     0              0            0
## 5                     0     0              0            0
## 6                     0     0              0            0
##   mobile_likes_received www_likes www_likes_received age_with_months
## 1                     0         0                  0        14.08333
## 2                     0         0                  0        14.08333
## 3                     0         0                  0        14.08333
## 4                     0         0                  0        14.00000
## 5                     0         0                  0        14.00000
## 6                     0         0                  0        14.00000
\end{verbatim}

\begin{center}\rule{0.5\linewidth}{\linethickness}\end{center}

\subsubsection{Age with Months Means}\label{age-with-months-means}

\begin{Shaded}
\begin{Highlighting}[]
\KeywordTok{library}\NormalTok{(dplyr)}

\NormalTok{age_with_months <-}\StringTok{ }\KeywordTok{group_by}\NormalTok{(pf, age_with_months)}
\NormalTok{pf.fc_by_age_months <-}\StringTok{ }\KeywordTok{summarize}\NormalTok{(}
\NormalTok{  age_with_months,}
  \DataTypeTok{friend_count_mean =} \KeywordTok{mean}\NormalTok{(friend_count),}
  \DataTypeTok{friend_count_median =} \KeywordTok{median}\NormalTok{(friend_count),}
  \DataTypeTok{n =} \KeywordTok{n}\NormalTok{()}
\NormalTok{)}

\NormalTok{pf.fc_by_age_months <-}\StringTok{ }\KeywordTok{arrange}\NormalTok{(pf.fc_by_age_months, age_with_months)}

\KeywordTok{head}\NormalTok{(pf.fc_by_age_months)}
\end{Highlighting}
\end{Shaded}

\begin{verbatim}
## # A tibble: 6 x 4
##   age_with_months friend_count_mean friend_count_median     n
##             <dbl>             <dbl>               <dbl> <int>
## 1            13.2              46.3                30.5     6
## 2            13.2             115.                 23.5    14
## 3            13.3             136.                 44.0    25
## 4            13.4             164.                 72.0    33
## 5            13.5             131.                 66.0    45
## 6            13.6             157.                 64.0    54
\end{verbatim}

\subsubsection{Noise in Conditional
Means}\label{noise-in-conditional-means}

\begin{Shaded}
\begin{Highlighting}[]
\KeywordTok{ggplot}\NormalTok{(}\KeywordTok{aes}\NormalTok{(}\DataTypeTok{x =}\NormalTok{ age_with_months, }\DataTypeTok{y =}\NormalTok{ friend_count_mean), }\DataTypeTok{data =} \KeywordTok{subset}\NormalTok{(pf.fc_by_age_months, age_with_months}\OperatorTok{<}\DecValTok{71}\NormalTok{)) }\OperatorTok{+}
\StringTok{  }\KeywordTok{geom_line}\NormalTok{()}
\end{Highlighting}
\end{Shaded}

\includegraphics{lesson4_student_files/figure-latex/Noise in Conditional Means-1.pdf}

\begin{center}\rule{0.5\linewidth}{\linethickness}\end{center}

\subsubsection{Smoothing Conditional
Means}\label{smoothing-conditional-means}

Notes:

\begin{Shaded}
\begin{Highlighting}[]
\KeywordTok{library}\NormalTok{(gridExtra)}
\end{Highlighting}
\end{Shaded}

\begin{verbatim}
## 
## Attaching package: 'gridExtra'
\end{verbatim}

\begin{verbatim}
## The following object is masked from 'package:dplyr':
## 
##     combine
\end{verbatim}

\begin{Shaded}
\begin{Highlighting}[]
\NormalTok{p1 <-}\StringTok{ }\KeywordTok{ggplot}\NormalTok{(}\KeywordTok{aes}\NormalTok{(}\DataTypeTok{x =}\NormalTok{ age, }\DataTypeTok{y =}\NormalTok{ friend_count_mean), }\DataTypeTok{data =} \KeywordTok{subset}\NormalTok{(pf.fc_by_age, age }\OperatorTok{<}\StringTok{ }\DecValTok{71}\NormalTok{)) }\OperatorTok{+}
\StringTok{  }\KeywordTok{geom_line}\NormalTok{() }\OperatorTok{+}
\StringTok{  }\KeywordTok{geom_smooth}\NormalTok{()}
\NormalTok{p2 <-}\StringTok{ }\KeywordTok{ggplot}\NormalTok{(}\KeywordTok{aes}\NormalTok{(}\DataTypeTok{x =}\NormalTok{ age_with_months, }\DataTypeTok{y =}\NormalTok{ friend_count_mean), }\DataTypeTok{data =} \KeywordTok{subset}\NormalTok{(pf.fc_by_age_months, age_with_months }\OperatorTok{<}\StringTok{ }\DecValTok{71}\NormalTok{)) }\OperatorTok{+}
\StringTok{  }\KeywordTok{geom_line}\NormalTok{() }\OperatorTok{+}
\StringTok{  }\KeywordTok{geom_smooth}\NormalTok{()}
\NormalTok{p3 <-}\StringTok{ }\KeywordTok{ggplot}\NormalTok{(}\KeywordTok{aes}\NormalTok{(}\DataTypeTok{x =} \KeywordTok{round}\NormalTok{(age }\OperatorTok{/}\StringTok{ }\DecValTok{5}\NormalTok{) }\OperatorTok{*}\StringTok{ }\DecValTok{5}\NormalTok{, }\DataTypeTok{y =}\NormalTok{ friend_count), }\DataTypeTok{data =} \KeywordTok{subset}\NormalTok{(pf, age }\OperatorTok{<}\StringTok{ }\DecValTok{71}\NormalTok{)) }\OperatorTok{+}
\StringTok{  }\KeywordTok{geom_line}\NormalTok{(}\DataTypeTok{stat =} \StringTok{'summary'}\NormalTok{, }\DataTypeTok{fun.y =} \StringTok{'mean'}\NormalTok{)}
\KeywordTok{grid.arrange}\NormalTok{(p1, p2, p3)}
\end{Highlighting}
\end{Shaded}

\begin{verbatim}
## `geom_smooth()` using method = 'loess'
\end{verbatim}

\begin{verbatim}
## `geom_smooth()` using method = 'loess'
\end{verbatim}

\includegraphics{lesson4_student_files/figure-latex/Smoothing Conditional Means-1.pdf}

\begin{center}\rule{0.5\linewidth}{\linethickness}\end{center}

\subsubsection{Which Plot to Choose?}\label{which-plot-to-choose}

Notes:

Make multiple plots during the exploritory phase and then refine them
down into the best plots for distribution.

\subsubsection{Analyzing Two Variables}\label{analyzing-two-variables}

Reflection:

Making multiple plots can show different features of the data. Also
while summaries and correlations are good for a lot of things they are
not always the best at portraying the data.

Click \textbf{KnitHTML} to see all of your hard work and to have an html
page of this lesson, your answers, and your notes!


\end{document}
